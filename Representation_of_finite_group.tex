\section{Representation of finite groups}
In this report $\F_q$ indicates the finite field with $q=p^r$ elements, where $p$ is a prime and $r$ is an integer greater then zero.

Here we are interested in doing Fourier analysis on some subgroups of the \emph{General linear group}, which is defined below. 
\begin{defn}
	The {\it General Linear group} of dimension $n$ over the field $\F_q$ is the group of $n\times n$ invertible matrices with entries in $\F_q$:
	\begin{equation*}
	\GL(n,\F_q)=\Big\{ A \in \F_q^{n \times n} \colon \det A = \abs A\neq 0 \Big\}.
	\end{equation*}	
\end{defn}
In particular we will focus on $GL(2,\F_q)$ and on its subgroup $\Aff(q)$, \emph{i.e.} the \emph{Affine group}:
\begin{defn}
	The {\it Affine group} of dimension two over the field $\F_q$ is:
	\begin{equation*}
	\Aff (q)=\Big\{ A=\begin{pmatrix} a & b \\ 0 & 1 \end{pmatrix}\in \GL(2,\F_q) \Big\}=\Big\{ A=\begin{pmatrix} a & b \\ 0 & 1 \end{pmatrix} \colon a,b \in \F_q, \, a \neq 0 \Big \}.
	\end{equation*}
\end{defn}
Since we want to do Fourier analysis on these groups, first of all we must map them (homomorphically) into groups of complex matrices. To do that we need to introduce the concept of representation of finite group $G$.

\begin{defn}
A (finite dimensional) representation of a finite group $G$ is a group homomorphism
\[
\pi:  G \rightarrow \GL(n,\C).
\]
If, for $g\in G$, $\pi(g)$ is a matrix with $i,j$ entry $\pi_{i,j}(g)$, we call the functions $\pi_{i,j} : G \rightarrow \C$ the \emph{matrix entries} of $\pi$.
\end{defn}
\begin{rem}
We can identify $\GL(n,\C)$ and
\[
\GL(V)=\{T\colon V\rightarrow V \ | \  T \text{ is linear and invertible}\},
\]
where $V$ is an $n$-dimensional vector space over $\C$. However notice that in this case the matrix entries $\pi_{i,j}$ change if change the basis of $V$ and this can cause some issue. In any case, will use the two concepts interchangeably.
\end{rem}

We give now some important definition about representations. In the following $G$ indicates a finite group.
\begin{defn}
\label{equiv:repres}
To representations $\alpha$ and $\beta$ of $G$ into $\GL(n,\C)$ are said to be \emph{equivalent}
if there exists a matrix $T \in \GL(n,\C)$ such that
\[
T\alpha(g)T^{-1}=\beta(g), \quad \forall g \in G.
\]
This is equivalent to say that we can obtain one representation from the other by a uniform change of basis.
\end{defn}

\begin{defn}
\label{def:unitary}
A \emph{unitary} representation is a representation $\pi$ which maps $G$ into the unitary group $\U(n)$, where 
\[
\U(n) =\{A\in \GL(n,\C) \colon {}^{t}\!\bar{A}A=I\}.
\]
\end{defn}
\begin{rem}
If $\Braket{u, \, v} = {}^t\!\bar{u}v$ is the standard Hermitian inner product of vectors of $\C^{n}$, the unitary matrices are those which preserve this product, that is, $\Braket{Au, \, Av} =\Braket{u, \, v} \ \forall u,v \in \C^{n}$.
Indeed, if $A \in \U(n)$, then by the definition of the inner product, we have $\Braket{Au, \, Av}={}^{t}\!\overline{(Au)}Av={}^t\!\bar{u}{}^t\!\bar{A}Av={}^t\!\bar{u}v=\Braket{u,\, v} \ \forall u,v \in \C^{n}$ and vice versa if $\Braket{Au, \, Av} =\Braket{u, \, v} \ \forall u,v \in \C^{n}$ then, by taking $u_i=(0,\dots ,0, 1, 0,\dots , 0)$ where the $1$ is in the $i$-$th$ coordinate, we see that, indicating with $a_i$ the $i$-$th$ column of $A$, $\delta_{i,j}=\langle u_i, \, u_j\rangle =\langle Au_i, \, Au_j\rangle={}^t\!\bar{a_i}a_j$, that is ${}^{t}\!\bar{A}A=I$.
\end{rem}
\begin{defn}
A \emph{subrepresentation} $\rho$ of a representation $\pi\colon G \rightarrow \GL(V)$ means that $\rho\colon G \rightarrow \GL(W)$, where $W$ is a subspace of $V$ such that $\pi(g)W\subset W \text{ for all }g \in G$ and $\rho(g)$ is the restriction of $\pi(g)$ to $W$, \it{i.e.}
\[
\pi(g)_{\mkern 1mu \vrule height 2ex\mkern2mu W}= \rho(g)\ \forall g \in G
\]
Using matrix language, this means that there is a basis of $V$ such that $\pi(g)$ has the block form:
\[
\begin{pmatrix}
\rho(g) & \ast \ \\
0 & \ast \
\end{pmatrix}
\]
\end{defn}
\begin{defn}
A representation  $\pi$ is \emph{irreducible} if its only subrepresentations are $\pi$ itself and 0.
\end{defn}
When we deal with representations of a finite group $G$, we can just consider the unitary ones, due to the following result:
\begin{prop}
\label{unit:eqiv}
If $G$ is a finite group, every representation is equivalent to a unitary representation.
\end{prop}
\begin{proof}[Proof sketch.]
To prove this preposition we'll use the fact (which we aren't going to prove) that if a matrix $M$ leaves invariant some positive definite, Hermitian inner product $c(u,v)$ for $u,v \in \C^n$, then $M$ is conjugate to a unitary matrix.
So it suffices to find an inner product which is invariant under $\pi(g)$ for all $g \in G$. Such a inner product is given by 
\[
c(u,v)= \sum_{g \in G} \Braket{ \pi(g)u, \pi(g)v } , \text{ for } u,v \in \C^n.
\]
Indeed, for $h\in G$, $u,v\in \C^n$, we have
%\begin{fleqn}
\[
\begin{split}
c(\pi(h)u,\pi(h)v) &= \sum_{g \in G} \Braket{ \pi(g)\pi(h)u, \pi(g)\pi(h)v } \\
&=\sum_{g \in G} \Braket{ \pi(gh)u, \pi(gh)v } \\
&=\sum_{k \in G} \Braket{ \pi(k)u, \pi(k)v }= c(u,v).\\
\end{split}
\]
%\end{fleqn}
Where we have made the substitution $gh=k$, so that if $h$ runs through all the elements in $G$, so does $k$. 
\end{proof}
Let us now introduce  the space of functions from $G$ to $\C$ 
\begin{equation*}
\LL^2(G)=\Set{f\colon G \rightarrow \C},
\end{equation*}
which is a vector space over $\C$ of dimension $\abs G$ (it is isomorphic to $\C^n$, with $n=\abs G$).
We can make $\LL^2(G)$ an algebra by defining the \emph{convolution} $a\ast b$, for $a,b\in \LL^2(G)$, $x\in G$:
\begin{equation*}
(a\ast b)(x)=\sum_{t \in G} a(xt^{-1})b(t)=\sum_{y \in G} a(y)b(y^{-1}x)
\end{equation*}
\begin{rem}[non necessario]
This definition of convolution coincides with that given for the commutative group $\Z/{n\Z}$ ($(a\ast b)(x)=\sum_{y \in \Z/{n\Z}} a(y)b(x-y)$). 
Moreover we have that, given $f\in \LL^2(G)$,
\begin{equation}
\label{conv:comm}
f*h=h*f \quad \forall h \in \LL^2(G)
\end{equation}

if and only if $f$ is constant on conjugacy classes
\[
\Set{g}=\Set{xgx^{-1}\colon x\in G}.
\]
Indeed if $f$ is constant on conjugacy classes, then for all $x\in G$
%\[
\begin{multline*}
(f\ast h)(x) =\sum_{t \in G} f(xt^{-1})h(t)=\sum_{xt \in G} f(x(xt)^{-1})h(xt)\\
=\sum_{xt \in G} f(xt^{-1}x^{-1})h(xt)
=\sum_{xt \in G} f(t^{-1})h(xt)
=\sum_{y \in G} f(y^{-1}x)h(y)= (h*f)(x),
\end{multline*}
%\]
and, conversely, if \ref{conv:comm} holds, then, taking $h=\delta_k$, for $k\in G$, we have that $(f\ast h)(ky)=f(kyk^{-1})$ and $(h*f)(ky)=f(y)$ for all $y \in G$, so that $f$ is constant on conjugacy classes.
Since if $G$ is commutative the conjugacy class $\{g\}$ contains only $g$, it follows that the convolution is commutative for abelian groups, as we already know.
\end{rem}
\begin{exmp}
We define the \emph{Left Regular Representation} $\LL$ of $G$ by 
\begin{align*}
\LL\colon G &\rightarrow \GL(\LL^2(G))\\
		 g&\mapsto \LL(g) 
\end{align*}
where 
\[
[L(g)a](x)=a(g^{-1}x)
\]
for $a \in \LL^2(G)$ and $x,g \in G$.
We have to check that the function L defined above is indeed a group representation, that is, we have to check $\LL(xy)=\LL(x)\LL(y)$ for all $x,y \in G$. Let's take $a \in \LL^2(G)$ and $g\in G$, then
\[ [\LL(xy)a](g)=a((xy)^{-1}g)=a(y^{-1}x^{-1}g)
\]
and
\[
[\LL(x)\LL(y)a](g)=[\LL(x)(\LL(y)a)](g)=[\LL(x)b](g)=b(x^{-1}g)=a(y^{-1}x^{-1}g),
\]
where $b=\LL(y)a \in \LL^2(G)$.

[Matrice di L??]
\end{exmp}

Now we want to decompose a representation $\pi$ of $G$ into irreducible part, \it{i.e.}, considering $\pi$ as a matrix, we want to replace $\pi$ by a matrix which has diagonal blocks that cannot be further decomposed. This is achieved by the following proposition.
Note that by \ref{unit:eqiv} we can assume all representations are unitary.
\begin{prop}[Complete Reducibility Theorem] The following hold:
\begin{enumerate}
\item Suppose that $\rho\colon G\rightarrow \U(m)$ is a subrepresentation of the representation $\pi\colon G\rightarrow \U(n)$. Then $\pi$ is equivalent to the representation with block diagonal form:
\[(\rho \oplus \sigma)(g)=
\begin{pmatrix}
\rho(g) & 0 \\
0 		& \sigma(g)
\end{pmatrix}.
\]
\item By induction, $\pi(g)$ is equivalent to
\[
(\pi_1(g)\dots \pi_r(g))= 
\begin{pmatrix}
\pi_1(g)  &{} &0\\
{} &\ddots &{} \\
0 &{}  &\pi_r(g)
\end{pmatrix},
\]
where each subrepresentation $\pi_i$  is irreducible. we say, when this happens, that $\pi$ is \emph{completely reducible}. 
\end{enumerate}
\end{prop}
\begin{proof}
\begin{enumerate}
\item We know that, by the remark after definition \ref{def:unitary}, if $\pi(g)$ is unitary then it leaves invariant the standard inner product $\braket{u,\, v}={}^t\!\bar{u}v$ for $u,v \in \C^n$. Moreover, since $\rho$ is a subrepresentation of $\pi$, by definition there is a subspace $W$ of $V=\C^n$ such that $\pi(g)W\subset W$ and, for all $g\in G$, $\pi(g)_{\mkern 1mu \vrule height 2ex\mkern2mu W}= \rho(g)$.  Consider the \emph{orthogonal complement space} $W^\perp$ of $W$:
\[
W^{\perp}= \Set{v \in V \colon \braket{u, \, v}=0 \text{ for all } w \in W}.
\]
Then we claim that $\pi(g)W^{\perp}\subset W^{\perp}$. Indeed, take $w\in W^{\perp}$, then, for all $v\in W$, we can write (since $\pi(g) \in \GL(n,\C)$) $v=\pi(g)v'$, where $v'=\pi(g)^{-1}v=\pi(g^{-1})v \in W$. Then we have 
\[
\braket{\pi(g)w,\, v}=\braket{\pi(g)w,\, \pi(g)v'}=\braket{w,\, v'}=0.
\]
And so $\pi(g)w\in W^{\perp}$. This means that the restriction $\pi(g)_{\mkern 1mu \vrule height 2ex\mkern2mu W^{\perp}}= \sigma(g)$ is a subrepresentation of $\pi$.  Then, since we are working in $\C^n$, we can just take a basis of $V$ consisting in a combination of a basis of $W$ and a basis of $W^{\perp}$. If we write $\pi(g)$ in this basis we obtain exactly the matrix given in the statement of the proposition.
\item This follows easily by induction on $n$.
\end{enumerate}
\end{proof}

The next lemma, due to Issai Schur, is crucial to do Fourier analysis on finite groups.

\begin{lemma}[Schur's Lemma].
Suppose that $\pi\colon G\rightarrow \GL(V)$ and $\rho\colon G\rightarrow \GL(W)$ are representations of $G$. We define the space $\I(\pi,\rho)$ of the \emph{intertwining operators} between $\pi$ and $\rho$ as follows
\[
\I(\pi,\rho)=\Set{L\colon V\rightarrow W | L \text{ is linear and } L\circ \pi(g)= \rho(g)\circ L , \ \forall g\in G}.
\]
Suppose $L\in \I(\pi,\rho)$, then the following hold:
\begin{enumerate}
\item The restriction of $\pi$ to $\ker(L)$ is a subrepresentation of $\pi$ and  the restriction of $\rho$ to $\Im(L)$ is a subrepresentation of $\rho$.
\item If $\pi$ and $\rho$ are both irreducible representations and $L$ is not the null map, then $L$ is an isomorphism between $V$ and $W$ (and so $\pi$ and $\rho$ are equivalent).
\item If $\pi$ is irreducible and $L\in \I(\pi,\pi)$,  then $\exists \lambda \in \C$ such that $Lv=\lambda v$, for all $v\in V$, \emph{i.e.}, $L=\lambda I$, where $I$ is the identity operator.
\item If both $\pi$ and $\rho$ are irreducible, then 
\[
\dim_\C\I(\pi, \rho)=
\begin{cases}
1 &\text{if $\pi$ and $\rho$ are equivalent,} \\
0 &\text{otherwise.}
\end{cases}
\]
\end{enumerate}
\end{lemma}
\begin{proof}
\begin{enumerate}
\item Let $v \in \ker(L)$, then 
\[
L(\pi(g)v)=\rho(g)(Lv)=\rho(g)0=0, \quad \forall g\in G,
\]
so that $\pi(g)v\in \ker(L)$.

Analogously if $w \in \Im(L)$, then $w=Lv$ for some $v\in V$, thus
\[
\rho(g)w=\rho(g)(Lv)=L(\pi(g)v)\in \Im(L), \quad \forall g\in G.
\]


\item Since $L\neq 0$, we have $\ker(L)\neq V$, so by 1. and the fact that $\pi$ is irreducible, this means $\ker(L)=\{0\}$. Analogously, since $L\neq 0$, we have $\Im(L)\neq 0$, that is, using again 1. and the fact that $\rho$ is irreducible, $\Im(L)=W$. So $L$ is bijective.
\item Since we are working on $\C$, there exist a root $\lambda$ of $\det(xI-L)$, that is, an eigenvalue of $L$. Denote with $V_\lambda$ the eigenspace corresponding to the eigenvalue $\lambda$:
\[
V_\lambda=\Set{v\in v \colon Lv=\lambda v}\neq \{0\}.
\] 
Then $\pi(g)(V_\lambda)\subset V_\lambda, since L(\pi(g)v)=\pi(g)(Lv)=\pi(g)(\lambda  v)=\lambda\pi(g) v$, and so $\pi(g)v\in V_\lambda$. This means that the restriction of $\pi$ to $V_\lambda$ is a subrepresentation of $\pi$, which is irreducible, thus $V_\lambda = V$ and this implies $L=\lambda I$ on $V$. 
\item Since every $\I(\pi,\rho)\ni L\neq 0$ is an isomorphism between $V$ and $W$ by point 2., if $\pi$ and $\rho$ are not equivalent, such an isomorphism cannot exist, and so $\I(\pi,\rho)=\{0\}$. Assume then that $\pi$ and $\rho$ are equivalent and consider $K,L\neq 0$ in  $\I(\pi,\rho)$. By point 2. $L$ is an isomorphism, so we can consider the map $M=L^{-1} \circ K\colon V\rightarrow V$. Since both $K$ and $L$ (and so $L^{-1}$) are nonnull, $M$ is a nonzero element of $\I(\pi,\pi)$. Then by 3. we have that there exists $0\neq \mu\in \C$ such that $M=\mu I$, \emph{i.e.} $K=\mu L$ and the dimension of $\I(\pi,\rho)$ is 1.
\end{enumerate}

\end{proof}
\begin{defn}
Let $\pi\colon G \rightarrow \GL(V)$ be a representation. We define the \emph{degree} $d_{\pi}$ (or dimension) of $\pi$ to be the dimension of the vector space $V$.
\end{defn}
\begin{cor}
Every irreducible representation of an abelian group $G$ must have degree 1.
\end{cor}
\begin{proof}
See \cite[p. 249]{terras_1999}.
\end{proof}
To do Fourier analysis on a finite group $G$ we will need a list of all inequivalent representation of $G$.

\begin{defn}
Let $G$ be a finite group, we denote with $\hat{G}$ a complete set of irreducible unitary representation of $G$.
\end{defn}
\begin{rem}
Note that every two one-dimensional representations must be inequivalent. This is because if $\pi$ is one-dimensional, then $\pi(G)$ is an element of $\C$ and so if $\rho$ is an equivalent representation, then by definition \ref{equiv:repres}, there exists $T\in \C$ such that $\rho(g)=T\pi(g) T^{-1} = TT^{-1}\pi(g)=\pi(g)$, for all $g\in G$.
\end{rem}
\begin{exmp}
$G=\Z/n\Z$ under addition modulo $n$.

In this case we have that $e_a(x)=\exp(2\pi iax/n)$ for $a,x\in G$ are one-dimensional representation of $G$, also called \emph{characters} of $G$. We will see (maybe) that $\hat{G}=\{e_a\colon a \in G\}$. 
\end{exmp}
\begin{defn}
Given a representation $\pi$ of $G$ we define the \emph{character} of $\pi$ as $\chi_\pi(g)=\text{Tr}(\pi(g))$, for all $g \in G$, where Tr($\pi(g)$) is the trace of the matrix $\pi(g)$.
\end{defn}
\begin{rem}
Note that $\chi_\pi(g)\colon G\rightarrow \C$, so if $\pi$ is one-dimensional, its character coincides with $\pi$ itself. Moreover, by definition we have $\chi_{\pi\oplus\rho}=\chi_{\pi} + \chi_\rho$.
\end{rem}