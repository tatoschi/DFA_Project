\section{Introduction}
This report is divided in two main parts: in the first, more abstract, we developed representation theory of finite 
(even non abelian) groups; in the second we studied some kind of graphs over the finite upper half plane, that is a subset of
certain finite fields. The main reference is our textbook \cite{terras_1999}, in particular chapters 
(mettere qui i capitoli che hai usato) for the part of representation theory and chapter 18 for the second part.
Our work was studying and filling the gaps (examples and proofs left as exercises) in these chapters. It was not easy,
since the exposition is very strict and many proofs are just sketched or omitted with a link to a reference. In any case,
especially in the first part, we had to skip some details and proofs for reason of space, in order to reach the 
theorems we needed for the second part.

Mettere qua le giustificazioni per cui rappresentazione di gruppi è interessante.

In the second part we had the chance to develop and apply the theory of finite fields
to a concrete example. This lead us to some concepts (e.g. absolutely irreducibility, number of solutions of certain equations)
that we can find also in the theory of elliptic curves. In fact we find an equation of the form $x^2=f(y)$.
Moreover we wrote a MAGMA program that allowed us to compute some characteristic of the graphs we where analyzing, to have 
the felling of what was happening, and we used SAGE to plot these graphs for small number of vertexes.