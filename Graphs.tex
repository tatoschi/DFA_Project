\section{Graphs in the finite upper half plane}
\subsection{Poincaré upper half plane}
\subsection{The finite upper half plane}
\begin{defn}
	An element $\gamma \in \F_q$ is a {\it square} if $\exists \, x \in \F_q \colon \gamma = x^2$.
\end{defn}
If $\delta$ is a non square element of $\F_q$, then the polynomial $x^2 - \delta$ has no solutions in $\F_q$.
Its splitting field is $\F_{q^2}$ and one of its roots will be denoted by $\rd$ (the other is $-\rd$). 

$\rd$ will play the same role of the imaginary unit $i$. Given $z=x+y\rd \in \F_{q^2}$ we define,
using the notation from complex analysis, the {\it real part} of $z$ as $\Re z = x$;
the {\it imaginary part} of $z$ as $\Im z = y$;
the {\it conjugate} of $z$ as $\bar{z} = x-y\rd$; the {\it norm} of $z$ as $\Norm z = z \, \bar{z}$;
the {\it trace} of $z$ as $\Trace z = z+\bar{z}$.
\begin{rem}
The norm and the trace above are the ones usually defined in theory of finite fields (in the special case of the field 
extension $\F_{q^2}$ over $\F_q$), because $z^q= (x+y\rd)^q = x^q+y^q \rd^q = x+y(-\rd) =\bar{z}$.
See for instance \cite{lidl1994introduction}.
\end{rem}
\begin{defn}
	The {\it finite upper half plane} is
	\begin{equation}
		H_q = \big\{ z=x+y \rd \colon x \in \F_q,\, y \in \F_q^* \big\} 
	\end{equation} 
\end{defn}

We recall the definition of group action.
\begin{defn}
  A {\it group action} of the group $G$ on the set $X$ is a map
  \begin{align*}
	\phi \colon H \times X & \longrightarrow X \\ (g,x) &\longmapsto \phi (g,x) = g \cdot x
\end{align*}
such that:
\begin{itemize}
\item $\forall \, x \in X,\, \iota \cdot x = x$, where $\iota$ denotes the identity of the group;
\item $\forall \, h,g \in G, \forall\, x \in X$ we have $(gh)\cdot x=g \cdot (h\cdot x)$.
\end{itemize} 
\end{defn}

In our case will have $X=H_q$, while $G$ will be the general linear group $\GL(2,\F_q)$ or its subgroup of
affine transformations $\Aff (q)$, defined below.
\begin{defn}
	The {\it General Linear group} of dimension two over the field $\F_q$ is:
	\begin{equation*}
	\GL(2,\F_q)=\Big\{ g=\begin{pmatrix} a_{1,1} & a_{1,2} \\ a_{2,1} & a_{2,2} \end{pmatrix} \colon a_{i,j} \in \F_q,\, \det g \neq 0 \Big\}
	\end{equation*}	
\end{defn}
\begin{defn}
	The {\it Affine group} of dimension two over the field $\F_q$ is:
	\begin{equation*}
	\Aff (q)=\Big\{ g=\begin{pmatrix} a & b \\ 0 & 1 \end{pmatrix}\in \GL(2,\F_q) \Big\}=\Big\{ g=\begin{pmatrix} a & b \\ 0 & 1 \end{pmatrix} \colon a,b \in \F_q, \, a \neq 0 \Big \}
	\end{equation*}
\end{defn}
Now we can define define the action we are interested in, and investigate some of its properties.
\begin{defn}\label{flt}
The group $\GL(2,\F_q)$ acts on $H_q$ by {\it fractional linear transformation}:
\begin{equation}
	\forall g= \begin{pmatrix} a & b \\ c & d \end{pmatrix}\in \GL(2,\F_q),\, \forall z \in H_q, \quad g \cdot z = \frac{az+b}{cz+d}.
\end{equation}
\end{defn}

We check the well definition of the action and find some properties in the following proposition.

\begin{prop}
Given $\cdot$ the action by fractional linear transformation, the following holds (with the same notations of \ref{flt}):
\begin{itemize}
\item[1.] $\Im (g\cdot z) = \frac{\Im z \det g}{\Norm (cz+d)}$ and
$\Re (g\cdot z) = \frac{ac\Norm z +bd+(ad+bc)\Re z}{\Norm (cz+d)}$;
\item[2.] the action is well defined;
\item[3.] the restriction of of the action to the subgroup $\Aff (q)$ is a {\it transitive} action, that is:
	$\exists \, \bar{z} \in H_q \colon \big( \forall z \in H_q\, \exists g \in \Aff (q)\,\,\text{such that}\,\, z=g \cdot \bar{z} \big) $. 
\end{itemize}
\begin{proof}
To prove 1. we have to show first that $\Norm (cz+d)\neq 0$. 
By definition of norm it suffices to prove that $cz+d\neq 0$  Let $z=x+y\rd$. Then
\begin{equation}\label{den}
	cz+d=0 \iff (cx+d)+(cy)\rd=0 \iff cy=0 \land cx+d=0 \iff c=d=0,
\end{equation}
but this cannot happen, because $\det g = ad-bc \neq 0$. Then is enough to perform some calculations.

Now we prove the second part. We already proved that $cz+d\neq 0$ in \ref{den}.
We have now to show that $g\cdot z \in H_q$, that is $\Im (g\cdot z)\neq 0$. But this follows from point 1:
$\Im (g\cdot z) = \frac{\Im z \det g}{\Norm (cz+d)}$, where $\Im z \neq 0$ because $z \in H_q$,
$\det g \neq 0$ because $g \in \GL(2,\F_q)$. (mancante: funziona bene col prodotto di matrici)

For the last part, take $\bar{z} = \rd \in H_q$. Then for any $z=x+y\rd \in H_q$ we have that $\begin{pmatrix} y & x \\ 0 & 1 \end{pmatrix} \cdot \bar{z} = z$ (note that $y\neq 0$ because $z \in H_q$, so the matrix defined belongs to $\Aff (q)$).
\end{proof}
\end{prop}
\begin{rem}
$\Aff q$ is a subgroup of $\GL(2,\F_q)$, so the action of $\GL(2,\F_q)$ is transitive too.
\end{rem}

Now we can introduce a {\it distance} which is analogous to the arch length in the Poicaré upper half plane.
\begin{defn}
	The {\it distance} of two elements of $H_q$ is defined by:
	\begin{align*}
	\dist \colon H_q \times H_q & \longrightarrow H_q \\ (z,w) & \longmapsto \dist (z,w)= \frac{\Norm (z-w)}{\Im z \Im w}= \frac{(x-u)^2-\delta (y-u)^2}{yu},
\end{align*}
where $z=x+y\rd$ and $w=u+v\rd$ (the definition is well posed because $z,w \in H_q \Rightarrow y,v \neq 0$).
\end{defn}
\begin{rem}
The {\it distance} defined above is not a {\it metric}, that is its image is in $\F_q$ and not in $\R$. No triangle inequality is possible. The only properties of metric we have are: 
\begin{itemize}
\item [1.] $\dist (z,w)=\dist (w,z)$;
\item [2.] $\dist (z,w) =0 \iff z=w$.
\end{itemize}
\begin{proof}
The first item is trivial by definition, while the second one requires more care.

If $z=w$ then $x=u \land y=v$, so $\dist (z,w) =0$.

If $\dist (z,w) =0$ then $(x-u)^2-\delta (y-v)^2=0 \Rightarrow (x-u)^2 = \delta (y-v)^2$.
If $y-v\neq 0$ then $\delta = ((x-u)(y-v)^{-1})^2$, but this is impossible because $\delta$ is a non square element.
So $y=v$, which implies $x=u$ and $z=w$.
\end{proof}
\end{rem}

We are interested in this distance because it is invariant under the action of the group $\GL(2,\F_q)$ on $H_q$.
\begin{prop}
Given $g \in \GL(2,\F_q)$ and $z,w \in H_q$, we have that $\dist (z,w)=\dist(g \cdot z,g\cdot w)$,
where $\cdot$ is the action by fractal linear transformation.
\begin{proof}
We first notice a property of the norm. Let $z\in H_q$, $a\in \F_q$. Then
\begin{equation}\label{norm_prop}
\Norm (az) = az \, \overline{az} = az\bar{a}\bar{z}=aza\bar{z}=a^2 z \bar{z}=a^2 \Norm z.
\end{equation}
Moreover, we recall that if $z,w \in \F_{q^2}^{*}$, then $\Norm (z\,w)= \Norm z \Norm w$.
Now we can prove the proposition. With the usual notation
	\begin{align*}
	\dist (g\cdot z,g\cdot w)&=\frac{\Norm (g\cdot z-g\cdot w)}{\Im (g\cdot z) \Im (g\cdot w)}
							  =\frac{\Norm(\frac{az+b}{cz+d} -
							   \frac{aw+b}{cw+d})}{\frac{\Im z \det g}{\Norm (cz+d)}\frac{\Im w \det g}{\Norm (cw+d)}}=\\
							  &=\frac{\frac{\Norm\big((az+b)(cw+d)-(aw+b)(cz+d)\big)}{\Norm(cz+d)\Norm (cw+d)}}
							  {\frac{\Im z \Im w (\det g)^2}{\Norm (cz+d)\Norm (cw+d)}}
							  =\frac{\Norm \big((z-w)(ad-bc)\big)}{\Im z \Im w (\det g)^2}=\\
							  &=\frac{(\det g)^2 \Norm (z-w)}{\Im z \Im w (\det g)^2}=\dist (z,w).
	\end{align*}
%We can prove the proposition in the case $z=\rd$, because the action is transitive: let $h$ be such that
%$h \cdot \rd = z$. Then $g^{'} \cdot z = g^{'} h \cdot \rd = g \cdot \rd$, where $g=g^{'} h$, and
%$g^{'} \cdot w^{'} = g^{'} h h^{-1} \cdot w^{'} = g^{'} h \cdot (h^{-1} \cdot w^{'})= g \cdot w$, where $w= h^{-1} \cdot w^{'}$.
\end{proof}
\end{prop}

\subsection{Graphs and their properties}
Now we can finally introduce the graphs we are interested in.
\begin{defn}
 Fix $0\neq a\in \F_q$. The graph $\xq a$ is defined with vertexes the elements of $H_q$,
 and with edges the pairs of vertexes $z,w\in H_q$ such that $\dist (z,w) = a$.
\end{defn}
\begin{rem}
We notice that the graph is undirected because $\dist (z,w) = \dist (w,z)$.
\end{rem}
(qui ci vanno gli esempi, sia ripresi dal libro che no)
(ci sono altre osservazioni carine sulle similutini col caso continuo, ma dovrei studiarci bene e non so se c'è spazio)

We can now state and prove one of the main theorems of the report.

\begin{theorem}
Let $q=p^r$ , where $p$ is an odd prime. Let $0\neq a\in \F_q$, and let $\delta \in \F_q$ be a non square element.
Then the following holds:
\begin{itemize}
\item[1.] if $a\neq 4\delta$ then the graph is $(q+1)-regular$;
\item[2.] if $0\neq c\in \F_q$ then the graphs $\xq a$ and $X_q(\delta c^2, a c^2)$ are isomorphic;  
\item[3.] $\forall 0 \leq s < r, s\in \N$ the graphs $\xq a$ and $X_q (\delta^{p^s}, a^{p^s})$ are isomorphic;
\item[4.] if $a\neq 4\delta$ then $\xq a$ is the Cayley graph for the group $\Aff q$ with generators:
\begin{equation}\label{generators}
	S_q(\delta,a) = \Big\{ \begin{pmatrix} y & x \\ 0 & 1 \end{pmatrix} \in \Aff q \colon x^2=ay+(y-1)^2 \delta \Big\};
\end{equation}
\item[5.] if $a\neq 4\delta$ then $\xq a$ is connected.
\end{itemize}


(ci sarebbe un altro punto, non mi sembra né utile né interessante, lo ometto per adesso)
\begin{proof}
1. WLOG I can assume $z=\rd$, because of the transitivity of the action and the invariance of the distance.
In fact, if $g \cdot \rd = z$, then 
\begin{equation*}
\lvert \{\bar{w} \colon \dist (\rd, \bar{w})=a\}\rvert=\lvert \{w \colon \dist (z,w)=a \} \rvert.
\end{equation*}
So we are interested in the solutions of the equation $\dist (\rd, z)=a$.

Suppose $z=x+y\rd$. Then
\begin{align} \label{distance}
	\dist (\rd, z)=a & \iff \Norm (\rd - z) = ay\\		\notag
					 & \iff (\rd - z)\overline{(\rd - z)}=ay\\ \notag
					 & \iff (-x+(1-y)\rd)(-x-(1-y)\rd )=ay\\   \notag
					 & \iff x^2 = ay + (y-1)^2 \delta. 
\end{align}
We must be careful: if $z$ is a solution of the previous equation, we must check that $z\in H_q$, that is $y\neq 0$.
But this is always true: if $y=0$ from the previous equation we obtain $\delta=(x(y-1)^{-1})^2$,
but this is impossible, since $\delta$ is a non square element.
So we can simply find solutions of $\Norm (\rd - z) = ay$ in $\F_{q^2}$, and they will belong to $H_q$ as well.

We recall (see \cite{lidl1994introduction}) that $\mathfrak{N}\colon \F_{q^2}^{*} \to \F_q^{*}$ is an onto group omomorphism.
We want to use this property to find the number of solutions of $\Norm (\rd - z) = ay$:
\begin{equation}
	\lvert \mathfrak{N}^{-1}(a)\rvert=\frac{\abs {\F_{q^2}^{*}}}{\abs {\F_q^{*}}}=\frac{q^2+1}{q-1}=q+1.
\end{equation}
Fix $c=(\frac{a}{2\delta}-1)\rd$ and $d=(1-\frac{a}{4\delta})a$. Then:
\begin{align*}
	\Norm (z+c) = d &\iff (z+c)(\bar{z}+\bar{c})=d \\
					&\iff ((z-\rd)+\frac{a}{2\delta})((\bar{z}+\rd)-\frac{a}{2\delta})=(1-\frac{a}{4\delta})a \\
					&\iff (z-\rd)(\bar{z}+\rd)+(\frac{a}{2\delta})(\bar{z}+\delta-z+\delta)+\frac{a^2}{4\delta}=a-\frac{a^2}{4\delta} \\
					&\iff \Norm (z-\rd)+(\frac{a}{2\delta})(-2y\rd +2\rd)=a \iff \Norm (z - \rd) = ay.
\end{align*}
So, if we set $w=z+c$, in the case $d\neq 0$, I can find $q+1$ solutions in $\F_{q^2}^{*}$ of $\Norm w = d$.
But $d=0 \iff a=0,4\delta$, cases that are excluded by hypothesis.
(il libro fa altre considerazioni oer concludere che il numero di soluzioni è precisamente q+1, a me pare suff così per concludere).

Hence the equation $\dist (\rd, z)=a$ has exactly $q+1$ solutions, that is the graph is $(q+1)-regular$.

2. We first notice that that $\delta c^2$ is a non square element, so the definition of the graph $X_q(\delta c^2, a c^2)$
is well posed. Clearly multiplying by $c^2$ is a bijection of $H_q$ to itself, so the vertexes of the two graphs are the same.

In the definition of $\Norm z = z \bar{z}$ seems that the choice of $\rd$ matters; but as we already noticed $\bar{z}=z^q$.
This means that $\Norm z = z z^q$ is independent from the choice of $\rd$. The same does not hold for the imaginary part:
if $z=x+y\rd$, then $z=x+yc^{-1} c \rd $. So, with an obvious notation, 
\begin{equation} \label{ims}
	\Im_{c\rd} (z) = c^{-1} \Im_{\rd} (z).
\end{equation}
What we need to prove can be stated as follows:
\begin{equation*}
	\frac{\Norm (z-w)}{\Im_{\rd} (z) \Im_{\rd} (w)}=\dist_{\rd} (z,w)=a \iff 
	\frac{\Norm (z-w)}{\Im_{c\rd} (z) \Im_{c\rd} (w)}=\dist_{c\rd} (z,w)=ac^2.
\end{equation*}
But this is easy thanks to \ref{ims}:
\begin{align*}
	\frac{\Norm (z-w)}{\Im_{c\rd} (z) \Im_{c\rd} (w)}=ac^2 &\iff \frac{\Norm (z-w)}{c^{-1}\Im_{\rd} (z)c^{-1} \Im_{\rd} (w)}=ac^2 \\
															&\iff \frac{\Norm (z-w)}{\Im_{\rd} (z) \Im_{\rd} (w)}=a.
\end{align*}

3. We can restrict the proof to the case $s=1$, the general statement follows by induction.
Raising to $p$ is a field automorphisim, hence non square elements are mapped into non square elements 
and it is a bijection from $H_q$ to itself. So the graph $X_q (\delta^{p}, a^{p})$ is well defined and the vertexes of 
$\xq a$ and $X_q (\delta^{p}, a^{p})$ are the same. We observe in particular that $(x+y\rd)^p=x^p+y^p\rd^p$.
Moreover, using a notation similar to the previous point,
\begin{align*}
	\dist_{\rd} (z,w)=a & \iff \frac{\Norm (z-w)}{\Im_{\rd} (z) \Im_{\rd} (w)}=a\\
	& \iff (\frac{\Norm (z-w)}{\Im_{\rd} (z) \Im_{\rd} (w)})^p = a^p \\
	& \iff \frac{\Norm ((z-w)^p)}{\Im_{\rd^p} (z^p) \Im_{\rd^p} (w^p)}=a^p\\
	& \iff \frac{\Norm (z^p-w^p)}{\Im_{\rd^p} (z^p) \Im_{\rd^p} (w^p)}=a^p \iff \dist_{\rd^p} (z^p,w^p)=a^p,
\end{align*}
and we conclude that the two graphs are isomorphic.

4. In the statement we identify $H_q$ and $\Aff q$ by the bijection given by the action on the point $\rd$:
$x+y\rd \leftrightarrow \begin{pmatrix} y & x \\ 0 & 1 \end{pmatrix}$.
Suppose $S_q(\delta,a)$ as in \ref{generators}. We notice that the equation in the definition of $S_q(\delta,a)$
is the same as in \ref{distance}. So we obtain $s \in S_q(\delta,a) \iff \dist (\rd, s \cdot \rd) = a$. Moreover
\begin{align*}
	g,h \in \Aff q \quad \text{are adjacent} &\iff \dist (h \cdot \rd, g \cdot \rd) = a \iff \dist ( \rd, h^{-1}g \cdot \rd) = a\\
	&\iff h^{-1}g \in S_q(\delta,a) \iff \exists \, s \in S_q(\delta,a)\colon g=hs,
\end{align*}
so $S_q(\delta,a)$ is a set of generators for the graph. We only need to check that it is closed under inversion (our graph is undirected):
but this follows from the fact that $\dist(\rd, s\cdot \rd)=\dist(s^{-1}\rd, s^{-1}s\cdot \rd)=\dist(\rd, s^{-1}\cdot \rd)$.
Hence $\xq a$ is a Cayley graph with generators $S_q(\delta,a)$.

5. I want to prove that $S_q(\delta,a)$ generates $\Aff q$, that is every $g \in \Aff q$ can be expressed by the product
of a finite number of elements of $S_q(\delta,a)$. We first observe that
\begin{equation*}
	\begin{pmatrix} a & b \\ 0 & 1 \end{pmatrix}=\begin{pmatrix} 1 & b \\ 0 & 1 \end{pmatrix} \begin{pmatrix} a & 0 \\ 0 & 1 \end{pmatrix},
\end{equation*}
so it enough to show that $\forall b\in \F_q$ and $\forall a \in \F_q^{*}$ the matrices 
$\begin{pmatrix} 1 & b \\ 0 & 1 \end{pmatrix}$ and $\begin{pmatrix} a & 0 \\ 0 & 1 \end{pmatrix}$
belongs to the subgroup generated by $S_q(\delta,a)$.

Now, since $\vert S_q(\delta,a)\vert=q+1$ (this is because its cardinality is exactly the number of points adjacent to $\rd$,
 and in the case $a\neq 4\delta$ the graph is $q+1$ regular), we have that 
\end{proof}
\end{theorem}



%
%The quintessential example is when R = Z , {\displaystyle R=\mathbb {Z} ,} {\displaystyle R=\mathbb {Z} ,} and S = { n } {\displaystyle S=\{n\}} {\displaystyle S=\{n\}} for some nonzero n ∈ Z {\displaystyle n\in \mathbb {Z} } n\in\mathbb{Z}. By "considering n {\displaystyle n} n to be zero", we are essentially reducing elements of Z {\displaystyle \mathbb {Z} } \mathbb {Z} to their remainder after dividing by n. Thus I = n Z = { . . . , − 2 n , − n , 0 , n , 2 n , . . . } {\displaystyle I=n\mathbb {Z} =\{...,-2n,-n,0,n,2n,...\}} {\displaystyle I=n\mathbb {Z} =\{...,-2n,-n,0,n,2n,...\}} and R / I = Z / n Z = Z n {\displaystyle R/I=\mathbb {Z} /n\mathbb {Z} =\mathbb {Z} _{n}} {\displaystyle R/I=\mathbb {Z} /n\mathbb {Z} =\mathbb {Z} _{n}} is the set of possible remainders after dividing an integer by n (ie. the cyclic group of order n).

